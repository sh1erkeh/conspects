\documentclass{article}

\usepackage{../definition}
\usepackage{../theorem}
\usepackage{../preamble}

\title{Лекция 10}
\author{}
\date{5 ноября 2024}

\begin{document}
\maketitle

\noindent Основные асимптотические формулы при \(x \to 0\):
\begin{enumerate}[leftmargin=1cm]
    \item \(\sin(x) = x + o(x)\)
    \item \(\displaystyle \cos(x) = 1 - \frac{x^2}{2} + o(x^2)\)
    \item \(\tan(x) = x + o(x)\)
    \item \(\displaystyle \log_a(x) = \frac{x}{\ln(a)} + o(x)\)
    \item \(a^x = 1 + x \cdot \ln(a) + o(x)\)
    \item \((1 + x)^\alpha = 1 + \alpha x + o(x)\)      
\end{enumerate}

\section*{Производные и дифференциал}

\begin{definition}
    Пусть \[\exists \lim\limits_{\Delta x \to 0} \frac{f(x + \Delta x) - f(x)}{\Delta x}\]
    Тогда эта величина называется производной функции \(f(x)\) и обозначается \(f^{\prime}(x)\).     
\end{definition}
\noindent
\textbf{Пример 1.} \(\displaystyle f(x) = const \implies f^{\prime} = \lim\limits_{\Delta \to 0} \frac{c - c}{\Delta x} = 0\).\\[0.15cm]
\textbf{Пример 2.} \(\displaystyle f(x) = x^n \implies f^{\prime} = \lim\limits_{\Delta x \to 0} \frac{(x + \Delta x)^n - x^n}{\Delta x} = \lim\limits_{\Delta x \to 0} \frac{n x^{n - 1} \cdot \Delta x + o(\Delta x)}{\Delta x} = n x^{n - 1}\).\\[0.15cm]
\textbf{Пример 3.} \(\displaystyle f(x) = \sin(x) \implies f^{\prime} = \lim\limits_{\Delta x \to 0} \frac{\sin(x + \Delta x) - \sin(x)}{\Delta x} = \lim\limits_{\Delta x \to 0} \frac{\displaystyle 2\sin\left(\frac{\Delta x}{2}\right)\cos\left(\frac{\displaystyle \Delta x}{2} + x\right)}{\Delta x} = \lim\limits_{\Delta x \to 0} \cos\left(\frac{\Delta x}{2} + x\right) = \cos(x)\).\\[0.15cm]
\textbf{Пример 4.} \(\displaystyle f(x) = \log_a{x} \implies f^{\prime} = \lim\limits_{\Delta x \to 0} \frac{\log_a(x + \Delta x) - \log_a(x)}{\Delta x} = \lim\limits_{\Delta x \to 0} \frac{\displaystyle \log_a\left(1 + \frac{\Delta x}{x}\right)}{\Delta x} = \frac{1}{x\ln(a)}\).

\section*{Односторонние производные}

\begin{definition}
    Пусть \[\exists \lim\limits_{\Delta x \to 0^+} \frac{f(x + \Delta x) - f(x)}{\Delta x}\]
    Тогда эта величина называется правосторонней производной функции \(f(x)\) и обозначается \(f^{\prime}_{+}(x)\). 
\end{definition}
\noindent
\underline{Замечание:} аналогично определяется левосторонняя производная \(f^{\prime}_{-}(x)\).

\begin{claim}
    Производная функции \(f(x)\) в точке \(x_0\) существует \(\iff \exists f^{\prime}_{+}(x_0) = f^{\prime}_{-}(x_0)\). 
\end{claim}
\noindent
\textbf{Пример 1.} \(f(x) = \left\vert x \right\vert\). \(\displaystyle f^{\prime} =
\begin{cases}
    \ \ 1 & \text{если}\ x > 0\\
    -1 & \text{если}\ x \leq 0\\  
\end{cases} \implies 
f^{\prime}_{+}(0) \neq f^{\prime}_{-}(0) \implies \nexists f^{\prime}(0)\).

\begin{definition}[Частная производная]
    Пусть дана функция \(f: \mathbb{R}^n \to \mathbb{R}\). Пусть
    \[
        \exists \lim\limits_{h_{i} \to 0} \frac{f(x_{1}, \dots , x_{i} - h_{i}, \dots , x_{m}) - f(x_{1}, \dots, x_{i}, \dots, x_{m})}{h_{i}} = a_{i}(x)
    \]
    Тогда эта величина называется частной производной функции \(f(x)\) в точке \(x = (x_{1}, \dots, x_{m})\) по переменной \(x_{i}\) и обозначается как \(\displaystyle \frac{\partial f}{\partial x_{i}}(x)\). 
\end{definition}

\section*{Геометрический смысл производной}

\begin{minipage}{0.35\textwidth}
    \begin{asy}
        import graph;
        import geometry;
        size(6cm);
                    
        xaxis("$x$", 0, 20, Arrow);
        yaxis("$y$", 0, 20, Arrow);

        real func(real x) { return 0.05 * (x - 4) * (x - 4) + 4.5; }
        draw(graph(func, 0, 20));

        real tangent_line(real x) { return 7 / 10 * x - 0.75; }
        draw(graph(tangent_line, 15 / 14, 20));
        
        pair T = (11, tangent_line(11)), B = (15 / 14, 0);

        markangle(Label("$\varphi_T = f^{\prime}(T_x)$", Relative(0.5)), n=1, (11, 0), B, T, blue);
        dot("$O$", (0, 0), S); dot("$T$", (11, 7.7 - 0.75), SE);
    \end{asy}
\end{minipage}%
\hfil
\begin{minipage}{0.6\textwidth}
    \begin{theorem}
        \(\exists f^{\prime}(x_0) \implies\) график \(f(x)\) имеет касательную в точке \(x_0\), причем угловой коэффициент касательной равен \(f^{\prime}(x_0)\). 
    \end{theorem}

    \begin{claim}
        Уравнение касательной к графку функции \(f(x)\) в точке \(x_0\) имеет вид \(y = (x - x_0) \cdot f^{\prime}(x_0) + f(x_0)\).   
    \end{claim}
\end{minipage}

\end{document}