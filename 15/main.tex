\documentclass{article}

\usepackage{../definition}
\usepackage{../theorem}
\usepackage{../preamble}

\title{Задача 15}
\date{}
\author{}

\begin{document}

\maketitle
\noindent\fbox{%
    \parbox{\textwidth}{
        \begin{enumerate}
            \item Покажите, что счетное произведение \(\mathbb{R}\) на себя имеет мощность континуум.
            \item Докажите, что мощность множества всех непрерывных функций \(f: \mathbb{R} \to \mathbb{R}\) равна континууму.
            \item Найдите мощность множества всех монотонных функций \(f: \mathbb{R} \to \mathbb{R}\). 
        \end{enumerate}
    }
}
\subsection*{Пункт 1}
\begin{enumerate}
    \item \(\mathbb{R} \sim \{0, 1\}^{\mathbb{N}} \implies\) существует биекция \(\varphi: \mathbb{R} \to  \{0, 1\}^{\mathbb{N}}\).
    \item Сопоставим последовательности \(\overline{x} = (x_1,\ x_2,\ x_3 \dots)\) действительных чисел последовательность \(\varphi(x_1),\ \varphi(x_2),\ \varphi(x_3) \dots\) 
    \item Получим биекцию \(\alpha: \mathbb{R}^{\mathbb{N}} \to (\{0, 1\}^{\mathbb{N}})^{\mathbb{N}}\) (так как \(\varphi\) --- биекция).
    \item Последовательности последовательностей сопоставим таблицу \(\Phi\), где \(\Phi_{ij} = \varphi(x_i)_j\).
    \item Получим биекцию \(\beta: (\{0, 1\}^\mathbb{N})^{\mathbb{N}} \to \{0, 1\}^{\mathbb{N} \times \mathbb{N}}\).
    \item \(\{0, 1\}^{\mathbb{N} \times \mathbb{N}} \sim \{0, 1\}^{\mathbb{N}} \implies \) существует биекция \(\gamma: \{0, 1\}^{\mathbb{N} \times \mathbb{N}} \to \{0, 1\}^{\mathbb{N}}\).
    \item Получена биеция: \(\displaystyle \mathbb{R}^\mathbb{N} \overset{\alpha}{\to} (\{0, 1\}^\mathbb{N})^{\mathbb{N}} \overset{\beta}{\to } \{0, 1\}^{\mathbb{N} \times \mathbb{N}} \overset{\gamma}{\to } \{0, 1\}^{\mathbb{N}} \overset{\varphi^{-1}}{\to} \mathbb{R}\). 
\end{enumerate}

\subsection*{Пункт 2}
\begin{enumerate}
    \item Покажем, что если значения непрерывных функций \(f_1,\ f_2: \mathbb{R} \to \mathbb{R}\) во всех рациональных точках совпадают, то эти функции тождественно равны.
    \item Функция \(f\) называется непрерывной в точке \(a\), если \(\exists \lim\limits_{x \to a} f(x) = f(a)\).
    \item Определение предела функции по Гейне: значение \(A\) является пределом функции при \(x \to a\), если для любой последовательности точек, сходящейся к \(a\), но не содержащей \(a\) в качестве одного из своих элементов, последовательность значений функции сходится к \(A\).
    \item \((2) \land (3) \implies (1) \implies \) непрерывная функция однозначно задана значениями в рациональных точках.
    \item Мощность множества всех функций \(f: \mathbb{Q} \to \mathbb{R}\) равна \(\left\vert \mathbb{R}^{\mathbb{Q}} \right\vert = \left\vert \mathbb{R}^{\mathbb{N}} \right\vert = \left\vert \mathbb{R} \right\vert\), ч.т.д.  
\end{enumerate}

\subsection*{Пункт 3}
\begin{enumerate}
    \item Рассмотрим все возрастающие функции \(f: \mathbb{Q} \to \mathbb{R}\). Мощность множетсва \(A\) всех таких функций равна \(\left\vert \mathbb{R} \right\vert\).
    \item Для произвольной функции \(f \in A\) поймем, сколько существует непрерывных функций \(g: \mathbb{R} \to \mathbb{R}\) таких, что во всех рациональных точках \(f(x) = g(x)\).
    \item Положим \(\displaystyle f^-(x) = \sup_{q \in \mathbb{Q} \cap {(-\infty,\ x)}} f(q)\), \(\displaystyle f^+(x) = \inf_{q \in \mathbb{Q} \cap {(x,\ +\infty)}} f(q)\). \(f^-(x) \leq g(x) \leq f^+(x)\).
    \item Положим \(B = \left\{x \in \mathbb{I}\ |\ (f^-(x),\ f^+(x)) \neq \varnothing \right\}\). Для разных \(x \in B\) соответствующие им интервалы не пересекаются \(\implies \left\vert B \right\vert = \left\vert \mathbb{N} \right\vert\).     
    \item Получим, что для произвольной функции \(f \in A\) сущетсвутет не более чем счетное количество непрерывных функций \(g: \mathbb{R} \to \mathbb{R}\) таких, что \(f(x) = g(x)\) во всех рациональных точках.
    \item Мощность искомого множества равна \(\left\vert \mathbb{R}^{\mathbb{N}} \right\vert = \left\vert \mathbb{R} \right\vert\).
\end{enumerate}

\end{document}
