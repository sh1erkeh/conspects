\documentclass{article}

\usepackage{../definition}
\usepackage{../theorem}
\usepackage{../preamble}

\title{Лекция 15}
\author{9 декабря 2024}
\date{}

\begin{document}
\maketitle

\section{Числовые последовательности}

\begin{definition}
    Числовая последовательность --- это функция натурального аргумента 
\end{definition}

\begin{definition}
    Число \(A\) называется пределом числовой последовательности \(x_n\), если
    \[
        \forall \varepsilon > 0\ \exists N \in \mathbb{N}: \forall n > N \implies \left\vert x_n - a \right\vert < \varepsilon 
    \]  
\end{definition}

\begin{theorem}[Вейерштрасса\noindent]
    \noindent Всякая монотонная ограниченная последовательность сходится.
\end{theorem}

\begin{theorem}
    \(\forall n\ a \leq x_n \leq b \) и \(\exists \lim\limits_{n \to \infty} x_n = C \implies C \in {[a, b]}\) 
\end{theorem}

\begin{definition}
    Стягивающейся системой отрезков называется последовательность отрезков \({[a_1, b_1]}, {[a_2, b_2]} \dots\) такая, что \(\forall n \in \mathbb{N}\ a_n \leq a_{n + 1} < b_{n + 1} \leq b_{n}\) и \( \lim\limits_{n \to \infty} (b_n - a_n) = 0\).  
\end{definition}

\begin{lemma}[О системе стягивающихся отрезков\noindent]
    \noindent Существует единственная точка, принадлежащая все отрезкам стягивающейся системы.  
\end{lemma}
\noindent \underline{Доказательство.}
\begin{enumerate}
    \item Существование
    \begin{enumerate}
        \item Заметим, что \(\{a_n\}\) возрастает, а \(\{b_n\}\) убывает. Кроме того, обе последовательности ограничены, так как \(a_1 \leq a_n < b_n \leq b_1\).
        \item По теореме Вейерштрасса \(\{a_n\}\) и \(\{b_n\}\) сходятся.
        \item Из теоремы \({(2)}\) следует, что \( \lim\limits_{n \to \infty } a_n = \lim\limits_{n \to \infty } b_n = c\).
        \item В силу монотонности \(\forall n \in \mathbb{N}\ a_n \leq c \leq b_n \implies \) точка \(c\) принадлежит всем отрезкам стягивающейся системы.
    \end{enumerate}
    \item Единственность
    \begin{enumerate}
        \item Предположим, что \(\exists d > c\), принадлежащая всем отрезкам стягивающейся системы.
        \item \(\forall n\ a_n \leq c < d \leq b_n \implies b_n - a_n \geq d - c > 0 \implies \lim\limits_{n \to \infty} (b_n - a_n) \geq d - c > 0\). Получено противоречие.
    \end{enumerate}    
\end{enumerate}
\noindent \underline{Замечание:} теорема иллюстрирует свойство непрерывности действительных чисел. Множество рациоальных чисел данным свойством не обладает.

\section{Предельные точки}

\begin{definition}[Подпоследовательность\noindent]
    \noindent Рассмотрим произвольные числовую последовательность \(\{x_n\}\) и возрастающую последовательность \(\{k_n\}\) натуральных чисел (\(k_n \geq n\)). Выберем из \(\{x_n\}\) члены \(x_{k_1}, x_{k_2}, \dots x_{k_n}, \dots \) Числовая последовательность \(\{x_{k_n}\}\) называется подпоследовательностью \(\{x_n\}\).  
\end{definition}
\noindent \underline{Замечание:} подследовательность является своей подпоследовательностью.

\begin{lemma}
    Последовательность \(\{x_n\}\) сходится к \(A \implies \) любая подпоследовательность \(\{x_n\}\) сходится к \(A\). 
\end{lemma}
\noindent \underline{Доказательство.}
\begin{enumerate}
    \item Зададим произвольный \(\varepsilon > 0\).
    \item Начиная с некоторого \(n\), все члены \(\{x_n\}\) лежат в \(\varepsilon\)-окрестности \(A\).
    \item Все члены \(x_{k_n}\) также будут лежать в \(\varepsilon\)-окрестности \(A\) в силу того, что \(k_n \geq n \implies  \lim\limits_{n \to \infty} x_{k_n} = A\). 
\end{enumerate}
\noindent \underline{Замечание:} у расходящейся последовательности могут быть сходящиеся подпоследовательности.

\begin{theorem}[Больцано-Вейерштрасса\noindent]
    \noindent Из любой ограниченной последовательности можно выделить сходящуюся подпоследовательность
\end{theorem}
\noindent \underline{Доказательство.}
\begin{enumerate}
    \item Последовательность ограничена \(\implies \exists a, b: \forall n\ a \leq x_n \leq b\).
    \item Разделим \({[a, b]}\) пополам, тогда по крайней мере один из получившихся отрезков содержит бесконечно много членов \(x_n\), обозначим его \({[a_1, b_1]}\). Пусть \(x_{k_1} \in {[a_1, b_1]}\). 
    \item Разделим отрезок \({[a_1, b_1]}\) пополам и обозначим через \({[a_2, b_2]}\) ту его половину, на которой лежит бесконечно много членов последовательности. Пусть \(x_{k_2} \in {[a_2, b_2]}\) и \(k_2 > k_1\).
    \item Будем продолжать данный процесс бесконечно долго, получим систему стягивающихся отрезков. При этом \(\forall n \in \mathbb{N}\ a_n \leq x_{k_n} \leq b_n\).
    \item По теореме о вложенных отрезках \(\exists! c \in {[a_n, b_n]} \implies \lim\limits_{n \to \infty} x_{k_n} = c\).
    \item Такми образом мы выделили подпоследовательность \(\{x_{k_n}\}\) исходной последовательности \(\{x_n\}\), которая сходится.   
\end{enumerate}
\noindent \underline{Замечание:} для неограниченных последовательностей теорема Больцано-Вейерштрасса неверна.\\
\textbf{Пример.} У возрастающей последовательности натуральных чисел не существует сходящихся подпоследовательностей.

\begin{definition}
    Последовательность \(\{x_n\}\) называется бесконечно большой, если
    \[
        \forall A > 0\ \exists N \in \mathbb{N}: \forall n > N \implies \left\vert x_n \right\vert > A
    \] 
\end{definition}
\noindent \underline{Замечание:} любая бесконечно большая последовательность является неограниченной. Обратное утверждение неверно (\(\{x_n\} = \{0, 1, 0, 2, 0, 3, \dots\}\)).\\
\textbf{Домашнее задание:} доказать, что из любой неограниченной последовательности можно выделить бесконечно большую.

\begin{definition}
    Число \(A\) называется предельной точкой последовательности \(\{x_n\}\), если из \(\{x_n\}\) можно выделить подпоследовательность, сходящуюся к \(A\).    
\end{definition}

\begin{definition}
    Число \(A\) называется предельной точкой \(\{x_n\}\), если в любой её окрестности содержится бесконечно много членов \(\{x_n\}\).  
\end{definition}

\begin{claim}
    Определения \({(6)}\) и \({(7)}\) эквивалентны.
\end{claim}

\begin{claim}
    Всякая ограниченная последовательность имеет по крайней мере одну предельную точку.
\end{claim}
\noindent \underline{Замечание 1:} если последовательность сходится, то предельная точка единственна.\\
\underline{Замечание 2:} последовательность может иметь сколько угодно предельных.

\begin{claim}
    \(\mathbb{Q}\) равномощно \(\mathbb{N}\) (счётно).  
\end{claim}

\noindent Каждое вещественное число из отрезка \({[0, 1]}\) является предельной для последовательности рациональных чисел отрезка \({[0, 1]}\).\\
\textbf{Домашнее задание.}
\begin{enumerate}
    \item Привести пример неограниченной последовательности, которая имеет ровно 2 предельные точки. 
    \item Привести пример ограниченной последовательности, которая имеет ровно 3 предельные точки. 
\end{enumerate}

\begin{definition}
    Пусть дана ограниченная последовательность \(\{x_n\}\).
    Наибольшая (наименьшая) предельная точка последовательности \(\{x_n\}\) называется верхним (нижним) пределом этой последовательности и обозначается как 
    \(\varlimsup\limits_{n \to \infty} x_n\ (\varliminf\limits_{n \to \infty} x_n)\). 
\end{definition}

\begin{claim}
    Последовательность \(\{x_n\}\) сходится \(\iff \varlimsup\limits_{n \to \infty} x_n = \varliminf\limits_{n \to \infty} x_n\).  
\end{claim}

\begin{theorem}
    Ограниченная последовательность имеет численные верхний и нижний пределы.
\end{theorem}
\noindent \underline{Замечание:} если последовательность \(\{x_n\}\)  не ограничена сверху (снизу), что говорят, что
\(\varlimsup\limits_{n \to \infty} x_n +\infty\ (\varliminf\limits_{n \to \infty} x_n = -\infty)\). 
\end{document}