\documentclass{article}

\usepackage{../definition}
\usepackage{../theorem}
\usepackage{../preamble}

\title{Лекция 2}
\author{}
\date{10 сентября 2024}

\begin{document}
    \maketitle
    \noindent
    \section{Модели вещественных чисел}
    Некоторые возможные модели действительных чисел:
    \begin{enumerate}
        \item Бесконечная периодическая десятичная дробь (\(\mathbb{R}_{a}\)) (Если вещественная дробь периодическая, то число является рациональным, если не периодическая -- иррациональное).
        \item Точка на числовой оси (\(\mathbb{R}_{b}\)).
    \end{enumerate}
    Обе реализации равноправны: между их обьектами можно установить биективное соответствие \(f: \mathbb{R}_{a} \to \mathbb{R}_{b}\),
    которое сохраняет арифметические операции и отношение порядка:
    \begin{itemize}
        \item \(f(x + y) = f(x) + f(y)\)
        \item \(f(xy) = f(x) \cdot f(y)\)
        \item \(x \leq y \Leftrightarrow f(x) \leq  f(y)\)
    \end{itemize}  
    Требования к аксиоматике:
    \begin{enumerate}
        \item Непротиворечивость.
        \item Независимость (ни одна аксиома не может быть получена как следствие из других аксиом).
        \item Полнота (между различными моделями можно установить изоморфизм).
    \end{enumerate}
    \section{Простейшие следствия аксиом сложения}
    \begin{theorem}
        В множестве \(\mathbb{R}\) существует единственный 0. 
    \end{theorem}
    \noindent
    \underline{Доказательство}.\\
    Предположим, что существуют два нуля \(0_{1}\) и \(0_{2}\). Докажем, что они всегда совпадают.
    По определению: \(0_{1} = 0_{1} + 0_{2} = 0_{2} + 0_{1} = 0_{2}\).
    \begin{theorem}
        В множестве \(\mathbb{R}\) у каждого элемента имеется единственный противоположный элемент. 
    \end{theorem}
    \noindent
    \underline{Доказательство}.\\
    Рассмотрим произвольный \(x \in \mathbb{R}\). Пусть \(x_{1}\) и \(x_{2}\) -- элементы противоположные x.
    Докажем, что \(x_{1} = x_{2}\). \(x_{1} = x_{1} + 0 = x_{1} + (x + x_{2}) = (x_{1} + x) + x_{2} = 0 + x_{2} = x_{2}\).
    \begin{theorem}
        Уравнение \(a + x = b\) в \(\mathbb{R}\) имеет (единственное) решение \(x = b + (-a)\).    
    \end{theorem}
    \noindent
    \underline{Доказательство}.\\
    Воспользуемся теоремой 2: \(a + x = b \Leftrightarrow (x + a) + (-a) = b + (-a) \Leftrightarrow x + (a + (-a)) = b + (-a)
    \Leftrightarrow\)\\ \(\Leftrightarrow x + 0 = b + (-a) \Leftrightarrow x = b + (-a)\).\\
    Замечание: выражение \(b + (-a)\) принято записывать как \(b - a\)(разность элементов b и a).\\
    \section{Простейшие следствия аксиом умножения}
    Аналогичные теоремы имеют место быть для операции умножения (доказательства этих теорем с точностью до замены символа и названия операции 
    дословно повторяют доказательства предыдущих трех теорем). 
    \begin{theorem}
        В \(\mathbb{R}\)  существует единственная единица.
    \end{theorem}
    \begin{theorem}
        Для каждого отличного от нуля элемента \(x \in \mathbb{R}\) существует единственный обратный элемент. 
    \end{theorem}
    \begin{theorem}
        Уравнение \(a \cdot x = b\) при любом \(a \neq 0\) имеет единственное решение \(b \cdot a^{-1}\).   
    \end{theorem}
    \section{Аксиома полноты (непрерывности), существование верхней (нижней) грани числового множества}
    \begin{definition}
        Некоторое подмножество \(X \subset \mathbb{R}\) ограничено сверху (снизу),
        если \(\exists c \in \mathbb{R}: \forall x \in X\ x \leq c\ (c \leq x)\). При этом число c называется верхней границей (мажорантой) (нижней границей (минорантой)) множества X.
    \end{definition}
    \begin{definition}
        Множество, ограниченное сверху и снизу, называется ограниченным.
    \end{definition}
    \noindent
    \underline{Замечание}: если множество ограничено сверху (снизу), то у него бесконечно много мажорант (минорант).
    \begin{definition}
        Элемент \(a \in X \subset \mathbb{R}\) называется наибольшим (максимальным) элементом \(X\), если
        \(\forall x \in X\ x \leq a\). 
    \end{definition}
    \noindent
    \underline{Замечание 1}: аналогично определяется наименьший (минимальный) элемент.\\
    \underline{Замечание 2}: максимальный (минимальный) элемент множества \(X\) обозначается как \(\max{X}\) (\(\min{X}\)).\\
    \underline{Замечание 3}: не во всяком (даже ограниченном) множестве имеется максимальный (минимальный) элемент. 
    \begin{definition}
        Наименьшее из чисел (обозначим его \(s\)), ограничивающих множество \(X\) сверху, называется верхней гранью (точной верхней гранью) \(X\) (обозначается как \(s = \sup{X}\)).  
    \end{definition}
    \noindent
    \underline{Замечание 1}: данное определение обозначает, что выполнены два условия:
    \begin{enumerate}
        \item \(s\) -- это одна из мажорант \(X\).
        \item \(\forall s^{\prime} < s\ \exists x^{\prime} \in X: s^{\prime} < x^{\prime} \) (то есть \(s\) не может быть уменьшено).
    \end{enumerate}
    \underline{Замечание 2}: аналогично определяется понятие точной нижней грани (\(i = \inf{X}\)).\\
    \underline{Домашнее задание}: записать точное определение инфимума.
    \section{Связь между супремумом и инфимумом, макисимумом и минимумом}
    \begin{claim}
        \(\sup{X} := \min\{c \in \mathbb{R}: \forall x \in X x \leq c\}\).
        \(\inf{X} := \max\{c \in \mathbb{R}: \forall x \in X x \geq c\}\).
    \end{claim}
    \underline{Замечание}: поскольку не всякое множество имеет максимальный (минимальный) элемент,
    необходимо доказать корректность сформулированных утверждений.
    \begin{theorem}[Принцип верхней грани]
        Всякое непустое ограниченное сверху подмножество \(\mathbb{R}\) имеет единственную точную верхнюю грань. 
    \end{theorem}
    \noindent
    \underline{Доказательство}.
    \begin{enumerate}
        \item Докажем единственность. В силу аксиомы антисимметричности (\((x \leq y) \land (y \leq x) \implies x = y\)) минимальный элемент может быть только один. 
        Тем самым осталось убедиться лишь в существовании верхней грани.
        \item Докажем существование. Пусть \(X \subset \mathbb{R}\), а множество \(Y = \{ y \in \mathbb{R}: \forall x \in X \implies (x \leq y)\}\) (множество верхних границ).
        По условию эти множества не пусты. Тогда по аксиоме полноты \(\exists c \in \mathbb{R}: \forall x \in X,\ \forall y \in Y \implies (x \leq c \leq y)\). Другими словами, такое число \(c\) (существование которого гарантировано аксиомой полноты) является для \(X\) мажорантой, а для \(Y\) минорантой. 
        Будучи мажорантой \(X\), \(c \in Y\); в то же время, как миноранта \(Y\), \(c = \min{Y} = \sup{X}\).
    \end{enumerate}
    \noindent
    \underline{Домашнее задание}: доказать аналогичное утверждение для инфимума.
\end{document}
